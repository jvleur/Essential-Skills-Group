\chapter{XML Internationalization}
Internationalization is the process of making software capable of being easily adaptable to different locations, technical requirements, etc. This includes for example making the language a modular part of the system, so that the system can easily be modified to show text in a different language.

Localization is the process of actually adapting such a piece of software to a specific locality. Localization can take place multiple times, for example for each language a GUI application is translated to.
\paragraph{i18n}
i18n is an abbreviation for internationalization, see following code.
\begin{minted}{bash}
>>> len('internationalization'[1:-1])
18
\end{minted}
L10n is the term used for \"localization\", following the same simple recipe.

Some companies like to add a their own abbreviations to the playing field, so terms like \"g11n\" (for globalization, which is internationalization and localization combined), \"world-readiness and localization\" and NLS (National/Native Language Support) can be found in the wild as well, basically referring to the same concept.
